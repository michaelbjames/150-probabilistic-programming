\documentclass[11pt,twocolumn]{article}

\title{150PP Final Project Proposal: \\
Abstracting for a latent and observable variable}
\author{Michael James}
\date{\today}

\begin{document}
   \maketitle
\section{Introduction}
Fun, as a language, has problems but it inspired a higher-order way of thinking about probabilistic models. The paper "A Model-Learner Pattern for Bayesian Reasoning" introduced a way to reuse and abstract probabilistic code--albeit with some obtuse F\# metaprogramming.

Many applications do not require the full abstraction power of four parameters, namely a hyperparameter, a prior, an input, and an output. Instead, questions we have sought to answer in this course have been largely of the form: \textit{Given that I saw X, what is the probability Y caused it?} Our solutions in Haskell using the probability monad and \texttt{pfilter} work and are only readable if the reader already knows what is going on. Perhaps by improving our \texttt{Distribution} kind with two parameters we can make both more readable and more expressive code. It may also be possible that with a more specialized kind, maybe with name and kind: $$\texttt{Bayesian :: * -> * -> *}$$ we can write spec implementations that will natively include improvements over previous class attempts. Simply put, with a designated latent and observed variable field, what happens?

\section{The Work}
I plan to create an API suite and simple implementation for this \texttt{Bayesian} kind. I plan to explore various type manipulations and determine what is immediately useful.

\section{The Goal}
I would like to produce a sensible API for latent and observed variables. Hopefully one that supports better code reuse than the \texttt{Distribution} kind.

\section{Stretch Goals}
If the API comes together quickly, then the next reasonable step is to write something with it. A reasonable something would be the dice-world problems because as a class, we have implementations and understanding of the problem with which to compare to my implementation.
\end{document}
